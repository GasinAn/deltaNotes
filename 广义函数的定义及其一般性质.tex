\chapter{广义函数的定义及其一般性质}

\section{测度}

\def\PRNotes{\href{https://github.com/GasinAn/PRNotes}{概率论笔记}}

阅读\PRNotes{}将有助于读者理解本章的内容.

任意集合上的测度是其子集到正实数的映射, 若把正实数替换为复数, 则称为复值测度\footnote{测度定义中的极限本来明确, 但如果映射的像为复数, 则极限必须明确为绝对收敛的极限.}.

设$\mu$为$\R^n$上一复值测度, 若$\varphi$为定义在$\R^n$上且在一个有界闭集外为$0$的复值函数, 则
\begin{equation}
    \mu(\varphi):=\int_{\R^n}\varphi\,\d\mu,
\end{equation}
现在我们也把函数$\mu(\cdot)$称为测度, $\mu(\cdot)$把函数映射成数, 所以是泛函.

定义在$\R^n$上的复值函数$\varphi$的支集为$\{\vec{x}\in\R^n\,|\,\varphi(\vec{x})\ne0\}$的闭包, 此集合等于$\{\vec{x}\in\R^n\,|\,\forall\,\vec{x}\,\text{的邻域}\,N,\varphi\,\text{不在}\,O\text{内为}\,0\}$, 也等于$\varphi$在其中为$0$的所有开集的并的余集.

设$\mu$为$\R^n$上一复值测度, $O\in\R^n$为开集, 若任意定义在$\R^n$上且在一个有界闭集外为$0$的复值连续函数$\varphi$, $\varphi$的支集含于$O$ $\Rightarrow$ $\mu(\varphi)=0$, 则称$\mu$在$O$内为$0$. 复值测度$\mu$的支集为$\{\vec{x}\in\R^n\,|\,\forall\vec{x}\,\text{的邻域}\,N,\mu\,\text{不在}\,O\text{内为}\,0\}$, 此集合也等于$\mu$在其中为$0$的所有开集的并的余集.

对任意在$\R^n$上所有有界闭集上可积的函数$f$, 可定义
\begin{equation}
    \mu_f(A)=\int_A f\,\d V,
\end{equation}
则
\begin{equation}
    \mu_f(\varphi)=\int_{\R^n} f\varphi\,\d V,
\end{equation}
于是有对应关系$f\leftrightarrow \mu_f$, 可干脆直接记成$\mu_f=f$, $\mu_f$和$f$的支集也是相等的, 所以测度是在$\R^n$上所有有界闭集上可积的函数的推广.

记$\mu_\delta:\varphi\mapsto\varphi(\vec{0})$, 则$\mu_\delta$就是$\delta$函数, 通常所写的$\int_{\R^n} \delta\varphi\,\d V=\mu_\delta(\varphi)$.

\section{广义函数}

不仅$\delta'$不能定义成测度, 一些常见函数(如$f(x)=1/x$)也无法对应于测度, 所以我们必须彻底放弃测度而把广义函数定义成泛函.

基本空间$K:=\{\varphi:\R^n\to\mathbb{C}\,|\,\varphi\,\text{为}\,\text{C}^\infty,\varphi\,\text{的支集有界}\}$, 连续\footnote{``连续''的含义此处略.}线性泛函$K\to\mathbb{C}$称为$\R^n$上的广义函数. 测度显然有对应的广义函数.
\begin{theorem}
    对$\R$, $K\ne\{\varphi\,|\,\forall x\in\R,\varphi(x)=0\}$.
\end{theorem}
\begin{proof}
    \begin{equation}
        \xi (x):=\begin{cases}
            e^{\frac{1}{\left\lvert x\right\rvert^2-1}}&\left\lvert x\right\rvert\leqslant1,\\
            0&\left\lvert x\right\rvert>1,
        \end{cases}
    \end{equation}
    $\xi\in K$.
\end{proof}

设$\eta$为$\R^n$上一广义函数, $O\in\R^n$为开集, 若任意$\varphi\in K$, $\varphi$的支集含于$O$ $\Rightarrow$ $\eta(\varphi)=0$, 则称$\mu$在$O$内为$0$. 广义函数$\eta$的支集为$\eta$在其中为$0$的所有开集的并的余集.

现在$\mu_\delta(\varphi)=\int_{\R^n} \delta\varphi\,\d V$只有$\varphi\in K$时有意义, 但我们要求$\varphi$的支集不一定有界. 若$\eta$为支集有界的广义函数, $f$为$\text{C}^\infty$, $\alpha\in K$且在$\eta$的支集的一有界闭邻域上为$1$, 则$\eta(\alpha f)$有定义且不依赖于$\alpha$, 故$\eta(\varphi):=\eta(\alpha\varphi)$.
\begin{theorem}
    $\mu_\delta(f)=f(\vec{0})$.
\end{theorem}
\begin{proof}
    任意不含$\vec{0}$的开集$O$, 任意$\varphi\in K$, $\varphi$的支集含于$O$ $\Rightarrow$ $O$的余集含于$\varphi$的支集的余集 $\Rightarrow$ $O$的余集含于$\varphi$在其中为$0$的所有开集的并 $\Rightarrow$ 在$O$的余集上$\varphi$为$0$ $\Rightarrow$ $\varphi(\vec{0})=0$ $\Rightarrow$ $\mu_\delta(\varphi)=0$. 任意含$\vec{0}$的开集$O$, 存在$\varphi\in K$, $\varphi(\vec{0})\ne0$ $\Rightarrow$ $\mu_\delta(\varphi)\ne0$ $\Rightarrow$ ($\varphi$的支集含于$O$ $\nRightarrow $ $\mu_\delta(\varphi)=0$). 所以$\mu_\delta$的支集为所有不含$\vec{0}$的开集的并的余集, 即$\{\vec{0}\}$, 因此$\mu_\delta$的支集有界. 设$\alpha\in K$在$\{\vec{0}\}$的一有界闭邻域上为$1$, 则$\mu_\delta(f)=\mu_\delta(\alpha f)=\alpha(\vec{0})f(\vec{0})=f(\vec{0})$.
\end{proof}

若$V$为矢量空间\footnote{此空间还需一些附加性质, 此处略.}, 则连续线性泛函$K\to V$称为矢量值广义函数.

若将基本空间定义中的$\R^n$换成$n$维流形, 则测度仍有唯一对应的广义函数, 但函数不一定有唯一对应的广义函数, 除非体元确定.
