% 编译方式: xelatex*2
\documentclass{ctexbook}
\usepackage{amsfonts}
\usepackage{amsmath}
\usepackage{amssymb}
\usepackage{amsthm}
\usepackage{mathrsfs}
\usepackage{hyperref}
\usepackage{syntonly}
\usepackage{IEEEtrantools}
%\syntaxonly
\pagestyle{plain}
\hypersetup{
    colorlinks,
    linkcolor=blue,
    filecolor=pink,
    urlcolor=cyan,
    citecolor=red,
}
\def\b{\boldsymbol}
\def\d{\mathrm{d}}
\def\p{\partial}
\def\R{\mathbb{R}}
\theoremstyle{definition}
\newtheorem{definition}{定义}[chapter]
\newtheorem{theorem}{定理}[chapter]
\title{广义函数论笔记}
\author{GasinAn}
\begin{document}
    \maketitle
    \noindent Copyright \textcopyright~2022 by GasinAn

\ 

\noindent All rights reserved. No part of this book may be reproduced, 
in any form or by any means, without permission in writing from the publisher, except by a BNUer.

\ 

\noindent The author and publisher of this book have used their best efforts
in preparing this book. These efforts include the development, research, and testing of the theories,
technologies and programs to determine their effectiveness.
The author and publisher make no warranty of any kind, express or implied,
with regard to these techniques or programs contained in this book.
The author and publisher shall not be liable in any event of incidental or consequential damages
in connection with, or arising out of, the furnishing, performance, or use of these techniques or programs.

\ 

\noindent Printed in China

    \tableofcontents
    \chapter{广义函数的定义及其一般性质}

\section{测度}

\def\PRNotes{\href{https://github.com/GasinAn/PRNotes}{概率论笔记}}

阅读\PRNotes{}将有助于读者理解本章的内容.

任意集合上的测度是其子集到正实数的映射, 若把正实数替换为复数, 则称为复值测度\footnote{测度定义中的极限本来明确, 但如果映射的像为复数, 则极限必须明确为绝对收敛的极限.}.

设$\mu$为$\R^n$上一复值测度, 若$\varphi$为定义在$\R^n$上且在一个有界闭集外为$0$的复值函数, 则
\begin{equation}
    \mu(\varphi):=\int_{\R^n}\varphi\,\d\mu,
\end{equation}
现在我们也把函数$\mu(\cdot)$称为测度, $\mu(\cdot)$把函数映射成数, 所以是泛函.

定义在$\R^n$上的复值函数$\varphi$的支集为$\{\vec{x}\in\R^n\,|\,\varphi(\vec{x})\ne0\}$的闭包, 此集合等于$\{\vec{x}\in\R^n\,|\,\forall\,\vec{x}\,\text{的邻域}\,N,\varphi\,\text{不在}\,O\text{内为}\,0\}$, 也等于$\varphi$在其中为$0$的所有开集的并的余集.

设$\mu$为$\R^n$上一复值测度, $O\in\R^n$为开集, 若任意定义在$\R^n$上且在一个有界闭集外为$0$的复值连续函数$\varphi$, $\varphi$的支集含于$O$ $\Rightarrow$ $\mu(\varphi)=0$, 则称$\mu$在$O$内为$0$. 复值测度$\mu$的支集为$\{\vec{x}\in\R^n\,|\,\forall\vec{x}\,\text{的邻域}\,N,\mu\,\text{不在}\,O\text{内为}\,0\}$, 此集合也等于$\mu$在其中为$0$的所有开集的并的余集.

对任意在$\R^n$上所有有界闭集上可积的函数$f$, 可定义
\begin{equation}
    \mu_f(A)=\int_A f\,\d V,
\end{equation}
则
\begin{equation}
    \mu_f(\varphi)=\int_{\R^n} f\varphi\,\d V,
\end{equation}
于是有对应关系$f\leftrightarrow \mu_f$, 可干脆直接记成$\mu_f=f$, $\mu_f$和$f$的支集也是相等的, 所以测度是在$\R^n$上所有有界闭集上可积的函数的推广.

记$\mu_\delta:\varphi\mapsto\varphi(\vec{0})$, 则$\mu_\delta$就是$\delta$函数, 通常所写的$\int_{\R^n} \delta\varphi\,\d V=\mu_\delta(\varphi)$.

\section{广义函数}

不仅$\delta'$不能定义成测度, 一些常见函数(如$f(x)=1/x$)也无法对应于测度, 所以我们必须彻底放弃测度而把广义函数定义成泛函.

基本空间$K:=\{\varphi:\R^n\to\mathbb{C}\,|\,\varphi\,\text{为}\,\text{C}^\infty,\varphi\,\text{的支集有界}\}$, 连续\footnote{``连续''的含义此处略.}线性泛函$K\to\mathbb{C}$称为$\R^n$上的广义函数. 测度显然有对应的广义函数.
\begin{theorem}
    对$\R$, $K\ne\{\varphi\,|\,\forall x\in\R,\varphi(x)=0\}$.
\end{theorem}
\begin{proof}
    \begin{equation}
        \xi (x):=\begin{cases}
            e^{\frac{1}{\left\lvert x\right\rvert^2-1}}&\left\lvert x\right\rvert\leqslant1,\\
            0&\left\lvert x\right\rvert>1,
        \end{cases}
    \end{equation}
    $\xi\in K$.
\end{proof}

设$\eta$为$\R^n$上一广义函数, $O\in\R^n$为开集, 若任意$\varphi\in K$, $\varphi$的支集含于$O$ $\Rightarrow$ $\eta(\varphi)=0$, 则称$\mu$在$O$内为$0$. 广义函数$\eta$的支集为$\eta$在其中为$0$的所有开集的并的余集.

现在$\mu_\delta(\varphi)=\int_{\R^n} \delta\varphi\,\d V$只有$\varphi\in K$时有意义, 但我们要求$\varphi$的支集不一定有界. 若$\eta$为支集有界的广义函数, $f$为$\text{C}^\infty$, $\alpha\in K$且在$\eta$的支集的一有界闭邻域上为$1$, 则$\eta(\alpha f)$有定义且不依赖于$\alpha$, 故$\eta(\varphi):=\eta(\alpha\varphi)$.
\begin{theorem}
    $\mu_\delta(f)=f(\vec{0})$.
\end{theorem}
\begin{proof}
    任意不含$\vec{0}$的开集$O$, 任意$\varphi\in K$, $\varphi$的支集含于$O$ $\Rightarrow$ $O$的余集含于$\varphi$的支集的余集 $\Rightarrow$ $O$的余集含于$\varphi$在其中为$0$的所有开集的并 $\Rightarrow$ 在$O$的余集上$\varphi$为$0$ $\Rightarrow$ $\varphi(\vec{0})=0$ $\Rightarrow$ $\mu_\delta(\varphi)=0$. 任意含$\vec{0}$的开集$O$, 存在$\varphi\in K$, $\varphi(\vec{0})\ne0$ $\Rightarrow$ $\mu_\delta(\varphi)\ne0$ $\Rightarrow$ ($\varphi$的支集含于$O$ $\nRightarrow $ $\mu_\delta(\varphi)=0$). 所以$\mu_\delta$的支集为所有不含$\vec{0}$的开集的并的余集, 即$\{\vec{0}\}$, 因此$\mu_\delta$的支集有界. 设$\alpha\in K$在包含$\{\vec{0}\}$的一有界闭集上为$1$, 则$\mu_\delta(f)=\mu_\delta(\alpha f)=\alpha(\vec{0})f(\vec{0})=f(\vec{0})$.
\end{proof}

若$V$为矢量空间\footnote{此空间还需一些附加性质, 此处略.}, 则连续线性泛函$K\to V$称为矢量值广义函数.

若将基本空间定义中的$\R^n$换成$n$维流形, 则测度仍有唯一对应的广义函数, 但函数不一定有唯一对应的广义函数, 除非体元确定.

    \chapter{广义函数的求导}

任意在$\R^n$上所有有界闭集上可积的函数$f$, 任意$\varphi\in K$,
\begin{equation}
    \int\frac{\p f}{\p x_k}\varphi\,\d x_k=-\int f\frac{\p \varphi}{\p x_k}\,\d x_k, 
\end{equation}
则
\begin{equation}
    \eta_{\frac{\p f}{\p x_k}}(\varphi)=-\eta_f(\frac{\p \varphi}{\p x_k}), 
\end{equation}
所以任意广义函数$\eta$的偏导数
\begin{equation}
    \frac{\p \eta}{\p x_k}(\varphi):=-\eta(\frac{\p \varphi}{\p x_k}).
\end{equation}
\begin{theorem}
    \begin{equation}
        H(x)=\begin{cases}
            0&x<0,\\
            1&x>0
        \end{cases}
    \end{equation}
    为Heaviside函数\footnote{$H(x)$在$0$处的定义无关紧要, 因为$\{0\}$为零测集, 以下类似情况皆如此.}, 其导函数为$\delta(x)$.
\end{theorem}
\begin{proof}
    $\eta_{\frac{\d H}{\d x}}(\varphi)=-\eta_H(\frac{d \varphi}{d x})=-\int_0^\infty\frac{d \varphi}{d x}\,\d x=\varphi(0)-\varphi(\infty)$, $\eta_{\delta}(\varphi)=\varphi(0)$, $\varphi(\infty)=0$.
\end{proof}
\begin{theorem}
    \begin{equation}
        \int \delta'(x)f(x)\,\d x=-f'(0).
    \end{equation}
\end{theorem}
\begin{proof}
    只需证$\delta'(x)$的支集有界, 这是平易的.
\end{proof}
\begin{theorem} 在$\R^3$中,
    \begin{equation}
        \vec{\nabla}^2\frac{1}{\left\lvert \vec{x}\right\rvert }=-\vec{\nabla}\cdot\frac{\vec{x}}{\left\lvert \vec{x}\right\rvert^3 }=-4\pi\delta(\vec{x}).
    \end{equation}
\end{theorem}
\begin{proof}
    \begin{align}
        \int\left[\vec{\nabla}^2\frac{1}{\left\lvert \vec{x}\right\rvert }\right]\varphi(\vec{x})\,\d V
        &=\sum_{i}\int\frac{\p }{\p x_i}\left[\vec{\nabla}\frac{1}{\left\lvert \vec{x}\right\rvert }\cdot\vec{e}_i\right]\varphi(\vec{x})\,\d V\\
        &:=-\sum_{i}\int\left[\vec{\nabla}\frac{1}{\left\lvert \vec{x}\right\rvert }\cdot\vec{e}_i\right]\frac{\p }{\p x_i}\varphi(\vec{x})\,\d V\\
        &=-\sum_{i}\int\frac{\p }{\p x_i}\left[\frac{1}{\left\lvert \vec{x}\right\rvert }\right]\frac{\p }{\p x_i}\varphi(\vec{x})\,\d V\\
        &:=\sum_{i}\int\left[\frac{1}{\left\lvert \vec{x}\right\rvert }\right]\frac{\p^2 }{\p x_i^2}\varphi(\vec{x})\,\d V\\
        &=\int\left[\frac{1}{\left\lvert \vec{x}\right\rvert }\right]\vec{\nabla}^2\varphi(\vec{x})\,\d V\\
        &:=\lim_{\epsilon\to 0}\int_{\left\lvert \vec{x}\right\rvert \geqslant\epsilon}\frac{1}{\left\lvert \vec{x}\right\rvert }\vec{\nabla}^2\varphi(\vec{x})\,\d V,
    \end{align}
    由第二Green公式(注意$\varphi$和$1/\left\lvert \vec{x}\right\rvert$在无穷远处为$0$),
    \begin{IEEEeqnarray}{rCl}
        \lim_{\epsilon\to 0}\int_{\left\lvert \vec{x}\right\rvert \geqslant\epsilon}\frac{1}{\left\lvert \vec{x}\right\rvert }\vec{\nabla}^2\varphi(\vec{x})\,\d V
        &=&\lim_{\epsilon\to 0}\int_{\left\lvert \vec{x}\right\rvert =\epsilon}\frac{1}{\left\lvert \vec{x}\right\rvert }\frac{\p}{\p\vec{n}}\varphi(\vec{x})\,\d S\\
        &&  \negmedspace{}-\lim_{\epsilon\to 0}\int_{\left\lvert \vec{x}\right\rvert =\epsilon}\varphi(\vec{x})\frac{\p}{\p\vec{n}}\frac{1}{\left\lvert \vec{x}\right\rvert }\,\d S\\
        &&  \negmedspace{}+\lim_{\epsilon\to 0}\int_{\left\lvert \vec{x}\right\rvert \geqslant\epsilon}\varphi(\vec{x})\vec{\nabla}^2\frac{1}{\left\lvert \vec{x}\right\rvert }\,\d V,
    \end{IEEEeqnarray}
    因为在$\{\vec{x}\in\R^3\,|\,\vec{x}\ne\vec{0}\}$内$\vec{\nabla}^2(1/{\left\lvert \vec{x}\right\rvert })=0$, 所以
    \begin{equation}
        \lim_{\epsilon\to 0}\int_{\left\lvert \vec{x}\right\rvert \geqslant\epsilon}\varphi(\vec{x})\vec{\nabla}^2\frac{1}{\left\lvert \vec{x}\right\rvert }\,\d V=0,
    \end{equation}
    取$\R^3$一有界闭子集$C$, $\varphi$为$\text{C}^\infty$, 则$\p\varphi/\p\vec{n}$在其中有最大值$M$, 所以
    \begin{align}
        \lim_{\epsilon\to 0}\int_{\left\lvert \vec{x}\right\rvert =\epsilon}\frac{1}{\left\lvert \vec{x}\right\rvert }\frac{\p}{\p\vec{n}}\varphi(\vec{x})\,\d S
        &\leqslant\lim_{\epsilon\to 0}\int_{\left\lvert \vec{x}\right\rvert =\epsilon}\frac{1}{\left\lvert \vec{x}\right\rvert}M\,\d S\\
        &=\lim_{\epsilon\to 0}4\pi M\epsilon=0,
    \end{align}
    而(注意$\vec{n}$指向原点)
    \begin{align}
        -\lim_{\epsilon\to 0}\int_{\left\lvert \vec{x}\right\rvert =\epsilon}\varphi(\vec{x})\frac{\p}{\p\vec{n}}\frac{1}{\left\lvert \vec{x}\right\rvert }\,\d S
        &=-\lim_{\epsilon\to 0}\int_{\left\lvert \vec{x}\right\rvert =\epsilon}\varphi(\vec{x})\frac{1}{\left\lvert \vec{x}\right\rvert^2 }\,\d S\\
        &=-\lim_{\epsilon\to 0}4\pi\varphi(\vec{x})=-4\pi\varphi(\vec{0}).
    \end{align}
\end{proof}

\end{document}
